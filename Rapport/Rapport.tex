\documentclass[10pt,a4paper]{article}
\usepackage[utf8]{inputenc}
\usepackage{amsmath}
\usepackage{amsfonts}
\usepackage{amssymb}
\usepackage{graphicx}
\usepackage{fancyhdr}
\pagestyle{fancy}
\fancyhead{}
\fancyhead[LO,RE]{\slshape \leftmark}


\begin{document}
\title{Enhancements for Monte Carlo Tree Search in The Mario AI Framework}
\date{May 22, 2013}
\author{Emil Juul Jacobsen\\\texttt{ejuu@itu.dk}        
        \and Rasmus Greve\\\texttt{ragr@itu.dk}}
%        \and \emph{Supervisor:}\\Julian Togelius\\\texttt{juto@itu.dk}}
\maketitle
\thispagestyle{empty} %Removes default pagenumber from frontpage
\begin{abstract}
(Bare copy/paste-ish fra projektbasen, skal skrives om!)
In this experiment we explore different implementations and enhancements of the Monte Carlo Tree Search algorithm for an AI, in order to evaluate their performance and results in the Super Mario AI Benchmark tool. 
We have implemented the basic MCTS algorithm in the Mario AI 
Framework and characterised the performance and identification of 
the strengths and weaknesses of the algorithm relative to the 
framework. We have identified a set of refinements and alterations of the algorithm 
and through implementation and evaluation of these individually we came up
with compositions that greatly increase the performance of the AI.
\end{abstract}
\clearpage

\section{Introduction}
In this experiment we explore different implementations and enhancements of the Monte Carlo Tree Search (MCTS) algorithm for an AI, in order to evaluate their performance and results in the Super Mario AI Benchmark tool. The MCTS algorithm has shown great results in various classical board games but has (to our knowledge) not been tested on a real-time physics-based game like Super Mario. Like some of the games that MCTS has proven effective in, Super Mario has a great branching factor of the state space but differs in that simulating actions is quite computationally heavy. These differences make several modifications of the core algorithm interesting for our experiment because they can help build the tree in a manner that uses the simulations more effectively.
\clearpage

\section{Background}
\begin{itemize}
\item Om MCTS \cite{mctssurvey}
\item Om UCB og UCT \cite{mctssurvey} måske også \cite{mspacman}
\item (kort!) Om The Mario AI Framework  \cite{mario}
\end{itemize}
\clearpage

\section{Approach and Improvements}
\subsection{Monte Carlo Tree Search with UCT}
Kilde \cite{mctssurvey}
\subsection{Domain knowledge}
\subsubsection{Limited actions}
Hvis vi er nødt til at begrænse ham til ikke at bruge $\downarrow$ til alle de andre implementationer skal det fremgå her!
\subsubsection{Hole detection}
Hvis vi er nødt til at bruge hulgenkendelse til alle de andre implementationer skal det fremgå her!
\subsection{Softmax Backup}
\begin{equation}\label{eq:softmax_equation}
exploitation = Q * maxReward + (1 - Q ) * averageReward
\end{equation}
We use equation \ref{eq:softmax_equation} to calculate the exploration part for the confidence of nodes
\subsection{Macro actions}
\cite{salesman}
\subsection{Heuristic Partial Tree Expansion Policy}
\subsection{Checkpoints}
\subsection{(Combination)}
\clearpage

\section{Results}
\renewcommand{\arraystretch}{1.5}
\begin{table}[h]
	\centering
	\begin{tabular}{| l | l |}
		\hline
		\textbf{Method} & \textbf{Score} \\ \hline
		Softmax backup q = 0 (UCT) & 34,162 \\ \hline
		Softmax backup q = $\frac{1}{8}$ & - \\ \hline
		Softmax backup q = $\frac{1}{4}$ & 34,387 \\ \hline
		Softmax backup q = $\frac{1}{2}$ & 34,147 \\ \hline
		Softmax backup q = 1 & 26,842 \\ \hline
	\end{tabular}
	\caption{Results of using Softmax backup with different q values}
	\label{tab:softmax_results}
\end{table}
\begin{table}[h]
	\centering
	\begin{tabular}{| l | l | l |}
		\hline
		\textbf{Method} & \textbf{Avg. number of nodes} & \textbf{Score} \\ \hline
		MCTS w/ UCT, limit = 0			& - & - \\ \hline
		MCTS w/ UCT, limit = 1			& - & - \\ \hline
		MCTS w/ UCT, limit = 2			& - & - \\ \hline
		MCTS w/ UCT, limit = 4			& - & - \\ \hline
		MCTS w/ UCT, limit = 8			& - & - \\ \hline
		MCTS w/ UCT, limit = 16			& - & - \\ \hline
		MCTS w/ UCT, limit = $\infty$	& - & - \\ \hline
	\end{tabular}
	\caption{Results of using UCT with a different limit for random moves}
	\label{tab:uct_results}
\end{table}
\begin{figure}[h]
\centering
\includegraphics[width=6cm]{img/Forfulgt}
\caption{Mario being followed}
\label{fig:followed}
\end{figure}
Her er noget mere tekst, reference til figur \ref{fig:followed}
\clearpage
\section{Conclusion}
\clearpage
\begin{thebibliography}{99}

\bibitem{mctssurvey}
  C. B. Browne, E. Powley, D. Whitehouse, S. M. Lucas, P. I. Cowling, P. Rohlfshagen, S. Tavener, D. Perez, S. Samothrakis, and S. Colton, “A survey of monte carlo tree search methods,” Computational Intelligence and AI in Games, IEEE Transactions on, vol. 4, no. 1, pp. 1–43, 2012.

\bibitem{civii}
S. R. K. Branavan, D. Silver, and R. Barzilay, “Non-linear monte-carlo search in civilization ii,” in Proceedings of the Twenty-Second international joint conference on Artificial Intelligence-Volume Volume Three, 2011, pp. 2404–2410.

\bibitem{salesman}
E. J. Powley, D. Whitehouse, and P. I. Cowling, “Monte Carlo Tree Search with macro-actions and heuristic route planning for the Physical Travelling Salesman Problem,” in Computational Intelligence and Games (CIG), 2012 IEEE Conference on, 2012, pp. 234–241.

\bibitem{mspacman}
T. Pepels and M. H. Winands, “Enhancements for Monte-Carlo Tree Search in Ms Pac-Man,” in Computational Intelligence and Games (CIG), 2012 IEEE Conference on, 2012, pp. 265–272.

\bibitem{mario}
S. Karakovskiy and J. Togelius, “The mario ai benchmark and competitions,” Computational Intelligence and AI in Games, IEEE Transactions on, vol. 4, no. 1, pp. 55–67, 2012.

\end{thebibliography}

\end{document}